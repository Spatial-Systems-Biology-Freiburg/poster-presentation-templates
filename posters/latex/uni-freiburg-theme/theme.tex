% Need for calculation of different length scales of title etc.
\usepackage[nomessages]{fp}
% Needed to define the new command for including the university logo
\usepackage{xparse}
% Needed to calculate the current position to stretch the last block
\usepackage{zref-savepos}
% Use to include uni logos
\usepackage{wrapfig}

% Define the colors to be used
\definecolorstyle{fdm-freiburg-colors} {
    \definecolor{darkblue}{HTML}{00004A}
    \definecolor{blue}{HTML}{344A9A}
    \definecolor{blue80}{HTML}{5D6BAD}
    \definecolor{blue60}{HTML}{868DC2}
    \definecolor{blue40}{HTML}{AFB1D8}
    \definecolor{blue20}{HTML}{D7D8EC}
}{
    % Background Colors
    \colorlet{backgroundcolor}{white}
    \colorlet{framecolor}{darkblue}
    % Title Colors
    \colorlet{titlefgcolor}{white}
    \colorlet{titlebgcolor}{darkblue}
    % Block Colors
    \colorlet{blocktitlebgcolor}{darkblue}
    \colorlet{blocktitlefgcolor}{white}
    \colorlet{blockbodybgcolor}{white}
    \colorlet{blockbodyfgcolor}{black}
    % Innerblock Colors
    \colorlet{innerblocktitlebgcolor}{white}
    \colorlet{innerblocktitlefgcolor}{black}
    \colorlet{innerblockbodybgcolor}{darkblue!30!white}
    \colorlet{innerblockbodyfgcolor}{black}
    % Note colors
    \colorlet{notefgcolor}{black}
    \colorlet{notebgcolor}{blue!50!white}
    \colorlet{noteframecolor}{blue}
}

% Define how long the title is on the left side
\ifdef{\titleleftrelativelength}{}{\def\titleleftrelativelength{0.8}}

\FPeval{\titlerightrelativelength}{1-\titleleftrelativelength}
\newlength{\titlewidthabsolute}
\newlength{\titlewidthleftabsolute}
\newlength{\titlewidthrightabsolute}
\setlength{\titlewidthabsolute}{811mm}
\setlength{\titlewidthleftabsolute}{\titleleftrelativelength\titlewidthabsolute - 30mm}
\setlength{\titlewidthrightabsolute}{\titlerightrelativelength\titlewidthabsolute - 25mm}


% Define a new titlestyle where FDM logo and university logo are included
\definetitlestyle{fdm-freiburg-title}{
    % A0 size = 841mm x 1189mm
    % Default margins are 15mm on both sides
    % => width = 841mm - 2*15mm = 811mm
    width=\titlewidthabsolute,
    % Use the same margin for top vertical space
    titletotopverticalspace=15mm,
    % Use slightly more margin for space to next blocks
    titletoblockverticalspace=15mm
}{
    \begin{scope}
        % Draw Rectangle for text of title
        \draw[color=blocktitlebgcolor, fill=titlebgcolor]
            (\titleposleft,\titleposbottom) rectangle (\titleposleft + \titleleftrelativelength\titleposright-\titleleftrelativelength\titleposleft - 10mm,\titlepostop);
        % Draw the logo of the university
        \node[below right, inner sep=0] (uni-fr-logo) at (\titleposleft + \titleleftrelativelength\titleposright-\titleleftrelativelength\titleposleft,\titlepostop)
            {\includegraphics
                [width=\titlerightrelativelength\titleposright-\titlerightrelativelength\titleposleft,
                height=\titlerightrelativelength\titleposright-\titlerightrelativelength\titleposleft]
                {uni-freiburg-theme/UFR-vorlage-designsystem-typo-farben-V1.910-1024x724-cropped.png}
            };
    \end{scope}
}

% Also adjust title centering and linebreaks
\settitle{
    \parbox{\titlewidthleftabsolute}{
        \vbox to \titlewidthrightabsolute {
            \color{titlefgcolor}
            {\bfseries \Huge \sc \@title \par}
            \vfil
            \vspace{1em}
            {\huge \@author \par}
            \vfil
            \vspace*{1em}
            {\Large \@institute}
            \vfil
        }
    }
}

% Define a new blockstyle which is simply a rectangle with small border
\defineblockstyle{fdm-freiburg-blockstyle}{
    titleleft,
    bodyverticalshift=5mm, linewidth=2pt,
    titleinnersep=6mm, bodyinnersep=0.2cm
}{
    \draw[color=framecolor, fill=blockbodybgcolor] (blockbody.south west)
        rectangle (blockbody.north east);
    \ifBlockHasTitle
        \draw[color=framecolor, fill=blocktitlebgcolor] (blocktitle.south west)
            rectangle (blocktitle.north east);
    \fi
}

% Include everything previously defined in a layout theme
\definelayouttheme{fdm-freiburg-theme}{
    \usecolorstyle{fdm-freiburg-colors}
    \usebackgroundstyle{Default}
    \usetitlestyle{fdm-freiburg-title}
    \useblockstyle{fdm-freiburg-blockstyle}
    \useinnerblockstyle{Default}
    \usenotestyle{Default}
}

% Hide watermark in bottom right corner
\tikzposterlatexaffectionproofoff

\def\footerminimumheight{4cm}

% Define a new command that will place the footer
\newcommand{\makefooter}[1]{
    \node [
        color=darkblue,
        above right,
        inner sep=0.4cm,
        minimum width=\paperwidth - 30mm,
        minimum height=\footerminimumheight,
        align=left,
        draw,
        fill=darkblue,
    ] at ([shift={(0.5*\pgflinewidth + 15mm,0.5*\pgflinewidth + 15mm)}]bottomleft) {
        \parbox{\paperwidth - 40mm}{
        % Save position of Footer. This parameter is used in the \stretchlastblock macro down below    
        \zsavepos{FooterPos}
        % Include the logos of FDM and University of Freiburg
        \begin{minipage}{\linewidth-25em}
            % Set color of text white
            \color{white}#1
        \end{minipage}%
        \begin{minipage}{25em}
            \colorbox{white}{\includegraphics[height=4em]{uni-freiburg-theme/fdm_logo.jpg}}
            \colorbox{white}{\includegraphics[height=4em, trim={-1.25em -0em -1.25em -0em}]{uni-freiburg-theme/wortmarke-grundform-universitaet_freiburg_blau__rgb.png}}
        \end{minipage}
        }
    };
}

% Define a new command that will include the Uni-Freiburg Logo
% This is meant to be used in the title of a block
\NewDocumentCommand{\unilogo}{O{uni-freiburg-theme/wortmarke-grundform-universitaet_freiburg_blau__rgb.png} D[]{height=1em, trim={-1.5em -0.25em -1.5em -0.25em}}}{
    \colorbox{white}{\includegraphics[#2]{#1}}
}

% Define new command that stretches the last block until it reaches the footer
% First define counter to differentiate between the different last-blocks
\newcounter{lastblocks}
\setcounter{lastblocks}{1}
% This is only a helper variable and will be overwritten continuously
\newlength{\difftofooter}

\NewDocumentCommand{\stretchlastblock}{}{
    % Save current position to file. This value will be read out in the second compile run
    \zsavepos{LastBlockPosition\arabic{lastblocks}}
    % Set the helper variable to current position
    \setlength{\difftofooter}{\zposy{LastBlockPosition\arabic{lastblocks}}sp}
    % Subtract the position of the footer
    \addtolength{\difftofooter}{-\zposy{FooterPos}sp}
    % Also subtract half a line for words
    \addtolength{\difftofooter}{-0.5em}
    % And subtract separation as defined above
    \addtolength{\difftofooter}{-15mm}
    % Create a vbox that expands the last block until it reaches the end
    \vspace{\difftofooter}
    % Increase the counter for next variable
    \addtocounter{lastblocks}{1}
}
